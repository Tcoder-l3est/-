%\documentclass{cumcmthesis}
\documentclass[withoutpreface,bwprint]{cumcmthesis} %去掉封面与编号页,电子版提交的时候使用。
\usepackage[framemethod=TikZ]{mdframed}
\usepackage{url}   % 网页链接
\usepackage{subcaption} % 子标题
\usepackage{algorithm} %排版算法
\usepackage{algorithmic} %排版算法
\usepackage[superscript]{cite}
\usepackage{enumitem} 

\addtocounter{MaxMatrixCols}{10}

\lstset{
    %numbers=left,                          	% 在左侧显示行号
    showstringspaces=false,        			% 不显示字符串中的空格
    frame=single,                         	% 设置代码块边框
}

\title{配电网可靠性和故障软自愈研究}
\tihao{C}
\baominghao{4321}
\schoolname{山东大学}
\membera{张朝阳}
\memberb{张荣凯}
\memberc{倪诗宇}
\supervisor{ }
\yearinput{2020}
\monthinput{08}
\dayinput{22}

\begin{document}

\maketitle

\begin{abstract}
    电网是全国民生、工业基础,由发、输、变、配、用各环节构成统一整体。运行需要由较高的稳定性保证供电质量,并且在时空拓扑结构不断变化的情况下,要对故障能够进行判断和处理。 利用图数据结构与实际电网在结构上保持一致,从而变成图论问题,自然、直观处理整体拓扑,并且能够在业务流动过程中,不断并入新的电网,实现业务拓展,图拓扑结构拓展。自然也就能够对可靠性进行图上分析,并且实现基于数据的智能化配电网自愈控制。\\
    \indent 针对问题一:本文采用模型+指标进行可靠性评估:首先给出评估指标,然后在图数据结构的基础下,使用等值最小路混合的方法,将元件可靠性分为最小路上以及非最小路上,并且最终邓加到最小路上,然后根据公式计算出可靠性。最后,对于拓扑信息进行了定性的灵敏度分析,给出了算例比较结果。\\
    \indent 针对问题二:本题根据电网的拓扑关系,首先推导出故障发生时电网元件之间动作的因果关系,构建因果网络和规则矩阵R。然后根据电网警告信息建立状态矩阵T,通过R的转置与T的计算得出故障信息向量T'。最后手动输入可能故障的节点F,通过T'与F相与,得到最终故障节点向量F',找到故障节点。\\
    \indent 针对问题三:本文首先将馈线以及开关进行图论抽象化,然后根据边权值检测推测出故障点,接着关闭邻接开关进行故障自动隔离,然后进行无故障段的恢复供电,通过连通分量个数的判断得出孤岛个数,以及边集的转化,实现孤岛的消除,进而完成负荷转移恢复供电。最后,给出了“软自愈”和“硬自愈”的成本差异估计。
    \keywords{最小路\quad 等值法 \quad 因果网络\quad 孤岛 \quad 连通分量\quad BFS}
\end{abstract}

\section{问题重述}
\subsection{问题背景}
配电网是供电系统的关键组成部分,是决定供电质量的重要因素,$90\%$
的断电都是因为配电网配置不合理导致的。配电网的可靠性是高质量供电的重要指标。连续时空拓扑信息,是一个利用发展的眼光研究配电系统业务的重要角度。而对于电网故障,故障自愈是一种事故恢复的快速自动化方法,能实现线路故障的自动判断、自动隔离和负荷转移,对于提升供电质量有很大的促进作用。
\subsection{问题提出}

{\bf 问题一}:针对配电网,建立一种基于连续时空电网拓扑的可靠性评估模型,并且要对连续时空拓扑信息差异引起的模型可靠性差异进行说明,也就是对拓扑信息进行灵敏度分析。\\
\indent {\bf 问题二}: 建立一种基于连续时空电网拓扑的配电网故障检测模型,并举例配电网信息差异引起的故障件检测差异。\\
\indent {\bf 问题三}: 基于“业务内生电网拓扑信息”,研究设计电网的“软自愈”方案,实现同类线路故障的自动判断、自动隔离、负荷转移恢复供电的算法,最后估算“软自愈”与“硬自愈”方案的成本差异,并且结合中压(10kv)电网故障图说明。

\section{问题分析}
\subsection{问题一分析}
要求建立可靠性评估模型,并且是依赖于连续时空电网拓扑的,已经明确了自变量。 对于可靠性评估,采用指标+模型的评估思路。先通过模型,分析拓扑信息,进一步量化指标,对可靠性进行一个合理评估。最后对自变量——电网拓扑信息进行灵敏度分析。

\subsection{问题二分析}
随着配电网的不断智能化,配电网可以通过SCADA系统\cite{yy5}进行数据监测和控制。通过收集和处理配电网的监测信息,运用故障发生时配电网中各种警告信息,能够逆向推断配电网的错误位置,进行配电网的故障检测。
\subsection{问题三分析}
软自愈实现方案:先将配电网抽象成图,之后根据异常电流值确定问题开关,
进而确定故障点,实现自动判断。然后将故障点周围开关关闭,使其孤岛化,
实现自动隔离。最后,对于因此产生的孤岛正常节点,
将其并入存在电源的连通分支,即实现了恢复供电。\, 估算与硬自愈的成本差异,


\section{模型假设}
\begin{assumption}
    假设配电网仅存在弱环电网或者无环拓扑结构。
\end{assumption}

\begin{assumption}
    配电网有完备的数据监控采集设备,能够根据记录的时间来判断监控设备开关动作的时序。可以用于查找故障主因,便于故障的后续分析。
\end{assumption}

\begin{assumption}
    配电网中包括三种类型的节点:故障节点、保护动作节点和短路器跳闸节点。\\
    \indent {\bf 例}: 故障节点线路L1上发生错误,则保护节点OP1将动作使断路器CB1跳闸。
\end{assumption}

\begin{assumption}
    此模型不讨论含有分布式电源DG的情况
\end{assumption}

\section{符号说明}
\begin{table}[!htbp]
    \caption{符号表}\label{tab:001} \centering
    \begin{tabular}{ccccc}
        \toprule[1.5pt]
        变量 & 说明 & 量纲 \\
        \midrule[1pt]
        $\lambda$ & 负荷点平均故障停运率 & 次/a \\
        $U$ & 负荷点平均停运时间 & h/a \\
        $SAIFI$ & 系统平均停电频率指标 & 次(户 $\cdot$ a) \\
        $SAIDI$ & 系统平均停电持续时间 & h(户$\cdot$ a) \\
        $ASAI$ & 平均供电可靠率 & \% \\
        $ASUI$ & 平均供电不可靠率 & \% \\
        $AIHC$ & 用户平均停电时间 & h \\
        $AITC$ & 用户平均停电次数 & 次 \\
        \bottomrule[1.5pt]
    \end{tabular}
\end{table}

\section{模型的建立与求解}
\subsection{问题一}
配电网的可靠性评估方法大致分为:
模拟法和解析法\cite{yy1}。解析法模型准确、
原理简单,并且便于对不同元件引起的拓扑结构差异
进行灵敏度分析,所以我们以解析法中的网络法为基础,
采用基于等值法和最小路法的混合算法评估模型\cite{yy2},
并且使用IEEE所提出的标准指标进行量化评价,
然后根据这些指标,对拓扑信息的灵敏度进行分析。

\subsubsection{指标建立}
配电网的可靠性指标分为负荷点和系统的可靠性指标。\\
\indent 负荷点的可靠性指标包括:1)负荷点平均故障停运率$\lambda$(次/a);2)负荷点平均停运时间$U$ (h/a);
系统的可靠性指标:系统平均停电频率指标$SAIFI$;系统平均停电持续时间$SAIDI$;平均供电可靠率$ASAI$;平均供电不可靠率$ASUI$;\\
\indent 针对电网的最基础串并联结构,我们给出相关指标计算公式:\\
\indent N个串联可修复元件,计算采用如下公式:
\begin{equation}
    \lambda = \sum_{i=1}^{n}\lambda_i
\end{equation}

\begin{equation}
    U = \sum_{i = 1}^{n} \lambda_i r_i
\end{equation}
\indent N个并联可修复元件,计算采用如下公式:
\begin{equation}
    \lambda = (\prod_{i = 1}^{n}\lambda_i)(\sum_{i = 1}^{n}r_i)
\end{equation}
\indent 对于系统可靠性指标,都可以通过负荷点的三个基本指标计算得到,计算采用如下公式:
\begin{equation}
    \left\{
        \begin{aligned}
            &SAIFI = \frac{\sum_{}^{}\lambda_iN_i}{\sum_{}^{}N_i} \quad ; \quad SAIDI = \frac{\sum_{}^{}U_iN_i}{\sum_{}^{}N_i} \\ 
            &ASAI = 1 - \frac{\sum_{}^{}U_iN_i}{8760\sum_{}^{}N_i} \quad ; \quad ASUI = 1 - ASAI
        \end{aligned}
    \right.
\end{equation}
\indent 公式中$\lambda_i^{'} r_i$为负荷点$i$的年平均故障率;$r_i$是负荷点$i$的年平均故障时间,$U$是负荷点的单次平均停运时间,$N_i$代表用户数。\\

\subsubsection{等值法简化电网}
    结构复杂的配电网,可以通过可靠性等值的方法\cite{yy3}将其等价为简单的辐射形配电网,简化拓扑信息,方便计算。
    实际配电系统一般由主副馈线组成,根据馈线数量分层处理,每一条馈线以及所连接的元件同属一层,之后等消除一条
    相应的等效分支线,自底而上逐层等价,最后得出简单拓扑结构的电网。\\
\indent 等值法的计算流程如图1(a)所示。
\begin{figure}[htbp]
    \centering
    \begin{minipage}[c]{0.48\textwidth}
        \centering
        \includegraphics[height=0.4\textheight]{1.png}
        \subcaption{等值算法流程}
        \label{g1}
    \end{minipage}
    \begin{minipage}[c]{0.48\textwidth}
        \centering
        \includegraphics[height=0.4\textheight]{2.png}
        \subcaption{最小路算法流程}
        \label{g2}
    \end{minipage}
    \caption{算法流程图}
    \label{gb}
\end{figure}


\subsubsection{Dijkstra最小路法}
配电网通常采用多闭环的设计思路,而运行过程中采用开环的辐射状结构。其次复杂配电系统等值之后,都可以简化为简单的辐射性网络。
作为可靠性评估,我们要看重不同元件的可靠度对负荷点的影响,从而进一步估计系统的影响。
所以,需要对元件进行分类,采用最小路法,把元件分成最小路元件以及非最小路元件,非最小路对负荷点的影响可以折算到最小路上。\\
 \indent   基本思路:将配电网抽象成图论上的无向图,
 具体图化方式为:将负荷母线以及变电站、变压器看作节点,线路、分段开关看作边,边权视为1,
 配电网可以抽象为图$\bf{G} = (V,E)$表示,自然得出图的邻接矩阵,电源看作最短路的源点,化成单源无向图。
 然后求每一个负荷点到电源最短路,即使用Dijstra求单源最短路。之后,把非最小路的元件故障影响折算到最小路,进行可靠性指标计算。
 Dijkstra最小路法算法流程如图1(b)所示。
    
\subsubsection{灵敏度分析}
配电网的可靠性水平与连续时空电网拓扑息息相关。而电网拓扑本质上是取决于电网的布线以及元件位置参数、性质参数,进而导致拓扑结构的位置不同。
因此,配电网的可靠性灵敏度分析是要对元件、设备的参数进行偏分,从而反应设备参数变化引起的时空拓扑变化对可靠性的影响程度。
使用参数\cite{yy4}用户平均停电时间$AIHC$以及用户平均停电次数$AITC$来计算灵敏度,公式如下:\\
\indent 用户平均停电时间对元件的故障率$\lambda$ 的灵敏度:
\begin{equation}
    \frac{\partial AIHC}{\partial \lambda_i} = \frac{\sum_{j\in i}r_{ij} N_j}{\sum_{j=1}^n N_j} 
\end{equation}
\indent 用户平均停电时间对元件的故障时间$r$ 的灵敏度:
\begin{equation}
    \frac{\partial AIHC}{\partial r_i} = \frac{\sum_{j\in i}\lambda_{ij} N_j}{\sum_{j=1}^n N_j} 
\end{equation}
\indent 用户平均停电次数对元件的故障率$\lambda$ 的灵敏度:
\begin{equation}
    \frac{\partial AITC}{\partial \lambda_i} = \frac{\sum_{j\in i} N_j}{\sum_{j=1}^n N_j} 
\end{equation}

\subsubsection{举例说明}
图2是一个较为普遍的辐射形配电网系统为例,以次结构说明配电网拓扑信息对于可靠性的影响以及灵敏度分析。
经过拓扑抽象图化后变为一个图论上的无向图,如图3所示。$V0$是电源点,$V2 \, V6 \, V7 \, V8$
分别对应负荷点$a \, b \, c \, d$,$V1 \, V3 \, V4 \, V5$是连接点,主馈线和分支线的交点。$E1-E8$是图的边。
\begin{figure}[H]
    \centering
    \includegraphics[width=12.5cm]{4.png}
    \caption{简单配电网拓扑图} %标题 , 显示在图片下面当图片的名字
\end{figure}

\begin{figure}[H]
    \centering
    \includegraphics[width=12.5cm]{3.png}
    \caption{抽象简化电网图} 
\end{figure}

\indent 首先对整个拓扑图,使用Dijstra算法,得到各个负荷点到电源$V0$的最短路长度以及路径,结果如表2。
\begin{table}[!htbp]
    \caption{最短路}\centering
    \begin{tabular}{ccccc}
        \toprule[1.5pt]
        顶点 & 最短跳数 & 路径 \\  
        \midrule[1pt]
        V1 & 1 & V0 $\rightarrow$ V1  \\
        V2 & 2 & V0 $\rightarrow$ V1 $\rightarrow$ V2 \\
        V3 & 2 & V0 $\rightarrow$ V1 $\rightarrow$ V3 \\
        V4 & 3 & V0 $\rightarrow$ V1 $\rightarrow$ V3 $\rightarrow$ V4 \\
        V5 & 4 & V0 $\rightarrow$ V1 $\rightarrow$ V3 $\rightarrow$ V4 $\rightarrow$ V5\\
        V6 & 3 & V0 $\rightarrow$ V1 $\rightarrow$ V3 $\rightarrow$ V6\\
        V7 & 4 & V0 $\rightarrow$ V1 $\rightarrow$ V3 $\rightarrow$ V4 $\rightarrow$ V7\\
        V8 & 5 & V0 $\rightarrow$ V1 $\rightarrow$ V3 $\rightarrow$ V4 $\rightarrow$ V5 $\rightarrow$ V8\\
        \bottomrule[1.5pt]
    \end{tabular}
\end{table}
为了简化计算,可以把元件的故障率都等效到边上,作为边的等效故障率。简化处理后的边故障信息如表3(a)所示。

\begin{table}
    \centering
    \begin{subtable}[t]{0.495\linewidth}
        \begin{tabular}{ccc}
            \toprule
            边 & 等效故障率 & 等效故障时间 \\  
            \midrule[1pt]
            E1 & 0.20 & 2小时 \\
            E2 & 0.10 & 5小时 \\
            E3 & 0.15 & 5小时 \\
            E4 & 0.20 & 5小时 \\
            E5 & 0.25 & 3小时 \\
            E6 & 0.20 & 3小时 \\
            E7 & 0.20 & 2小时 \\
            E8 & 0.10 & 1小时 \\
            \bottomrule
        \end{tabular}
        \caption{线路故障信息}
    \end{subtable}
    \begin{subtable}[t]{0.495\linewidth}
        \begin{tabular}{ccc}
            \toprule
            负荷点 & 平均故障率 & 平均故障时间 \\  
            \midrule[1pt]
            a & 0.30 & 0.90小时 \\
            b & 0.55 & 1.75小时 \\
            c & 0.75 & 2.55小时 \\
            d & 0.90 & 3.00小时 \\
            \bottomrule
        \end{tabular}
        \caption{负荷点可靠性指标}
    \end{subtable}
    \caption{拓扑信息}
    \label{tab:array}
\end{table}

之后,基于最短路首先进行负荷点可靠性指标评估,结果如表3(b)所示。再对系统的可靠性进行评估,结果如表5。

\begin{table}[!htbp]
    \caption{系统可靠性指标}\centering
    \begin{tabular}{ccccc}
        \toprule[1.5pt]
        $ASAI$\/\% & $ASUI$ \/\% & $SAIDI$\/(h\/(户·a)) & $SAIFI$\/(次\/(户·a)) \\  
        \midrule[1pt]
        99.9899 & 0.010073 & 0.882353 & 0.264706 \\
        \bottomrule[1.5pt]
    \end{tabular}
\end{table}
最后,基于不同的时空拓扑结构进行灵敏度的定性分析,
如图4的拓扑结构,配电网只有一条主干线路,多条分支线路,
没有分段开关,所以可以对8台配电变压器进行灵敏度分析,
基于拓扑结构,给出定性分析如表5。

\begin{table}[!htbp]
    \caption{灵敏度分析}\centering
    \begin{tabular}{ccccc}
        \toprule[1.5pt]
        影响排序 & 影响负荷点数量 & 负荷点编号 & 解释\\  
        \midrule[1pt]
        1 & 8 & a、b、c、d、f & 没有熔断器、断路器或其他保护装置\\
        2 & 1 & g & 有断路器,对其他元件影响弱\\
        3 & 1 & e & 有熔断器,对其他元件影响极微弱\\
        4 & 1 & h & 有熔断器和断路器,只影响本身\\
        \bottomrule[1.5pt]
    \end{tabular}
\end{table}

\begin{figure}[H]
    \centering
    \includegraphics[width=12.5cm]{5.png}
    \caption{基于不同元件拓扑的电网图} 
\end{figure}

\subsection{问题二}
通过收集的配电网中各种警告信息,来逆向推断故障原因。电网中的元件存在一定的拓扑关系,通过拓扑关系我们可以推导出其动作之间的因果关系,并且可以画出简单因果网络模型\cite{yy5},来描述电网中各元件之间的物理拓扑关系以及错误警告的依赖关系。通过警告信息,推断出可能存在错误的节点,结合实际问题分析对可能存在问题的节点进行筛选,最终得到故障节点。  

\subsubsection{建立规则矩阵}
依据 $R[i,j] = 1$,表明$P_j$是$P_i$发生的因,$P_i$是$P_j$发生的后果
\begin{equation}
        R[i,j] = 
        \begin{cases}
            1 , \quad P_j \Rightarrow P_i \\
            0 , \quad else
        \end{cases}
\end{equation}

\begin{figure}[htbp]
    \centering
    \includegraphics[width=10cm]{zrk-1.png}
    \caption{简单配电网络}
    \label{zrk-1}
\end{figure}

\begin{figure}[htbp]
    \centering
    \includegraphics[width=10cm]{zrk-2.png}
    \caption{因果关系网络}
    \label{zrk-2}
\end{figure}

\begin{table}[htbp]
    \caption{简单配电网络模型节点含义表}
    \label{zrk-3}
    \centering
    \begin{tabular}{cccc}
        \toprule[1.5pt]
        {\bf \text{节点}} &  {\bf \text{含义}} & {\bf \text{节点}} & {\bf \text{含义}}\\
        \midrule[0.5pt]
        $P_1$ & \text{线路$L_1$故障} & $P_7$ & \text{线路$L_2$故障} \\
        $P_2$ & \text{保护$OR_1$动作} & $P_8$ & \text{保护$OR_2$动作} \\
        $P_3$ & \text{断路器$CB_1$跳闸} & $P_9$ & \text{断路器$CB_2$跳闸} \\
        $P_4$ & \text{$OR_1$动作$CB_1$拒动} & $P_{10}$ & \text{$OR_2$动作$CB_2$拒动} \\
        $P_5$ & \text{保护$OR_3$动作} & $P_{11}$ & \text{母线$BUS$故障} \\
        $P_6$ & \text{断路器$CB_3$跳闸} &  & \\
        \bottomrule[1.5pt]
    \end{tabular}
\end{table}
根据上述定义,建立网络的规则矩阵

$$
    R = 
        \bordermatrix{
        % \begin{bmatrix}
              & P_1 & P_2 & P_3 & P_4 & P_5 & P_6 & P_7 & P_8 & P_9 & P_{10} & P_{11} \cr
          P_1 &  1 & 0 & 0 & 0 & 0 & 0 & 0 & 0 & 0 & 0 & 0 \cr
          P_2 &  1 & 1 & 0 & 0 & 0 & 0 & 0 & 0 & 0 & 0 & 0 \cr
          P_3 &  0 & 1 & 1 & 0 & 0 & 0 & 0 & 0 & 0 & 0 & 0 \cr
          P_4 &  0 & 0 & 0 & 1 & 0 & 0 & 0 & 0 & 0 & 0 & 0 \cr
          P_5 &  0 & 0 & 0 & 1 & 1 & 0 & 0 & 0 & 0 & 1 & 1 \cr
          P_6 &  0 & 0 & 0 & 0 & 1 & 1 & 0 & 0 & 0 & 0 & 0 \cr
          P_7 &  0 & 0 & 0 & 0 & 0 & 0 & 1 & 0 & 0 & 0 & 0 \cr
          P_8 &  0 & 0 & 0 & 0 & 0 & 0 & 1 & 1 & 0 & 0 & 0 \cr
          P_9 &  0 & 0 & 0 & 0 & 0 & 0 & 0 & 1 & 1 & 0 & 0 \cr
          P_{10} &  0 & 0 & 0 & 0 & 0 & 0 & 0 & 0 & 0 & 1 & 0 \cr
          P_{11} &  0 & 0 & 0 & 0 & 0 & 0 & 0 & 0 & 0 & 0 & 1 \cr
        % \end{bmatrix}
        }
$$

\subsubsection{建立输入向量}
警告信息向量T:反映了故障发生时,所有保护动作和断路器的动作信息。当某一结点$P_i$发生动作,系统应收到对应的警告信息,则T(i) = 1,否则T(i) = 0。
\begin{equation}
    T(i) = 
    \begin{cases}
        1 , \quad P_i\text{为真} \\
        0 , \quad \text{否则} \\
    \end{cases}
\end{equation}

可能故障元件向量F:依据实际情况,反映了可能存在故障的位置,用来缩小故障范围。
当$P_i$是可能存在故障的元件时F(i) = 1,否则F(i) = 0。
\begin{equation}
    F(i) = 
    \begin{cases}
        1 , \quad P_i\text{属于故障节点} \\
        0 , \quad \text{否则}
    \end{cases}
\end{equation}

\subsubsection{计算故障向量$T'$和 结果向量$F'$}
在规则矩阵$R$中$R[i,j] = 1$ ,表明$P_j$是$P_i$的因,$P_j$是$P_i$的果,对$R$进行转置,则矩阵$R^T$的第$j$行中所有为1的元素,表示以$P_j$为错误的因所能引发的结果。
依据输入的$T$,$T$中反应了电网故障的结果。$R^T$的第$j$行与$T$对应相乘,若结果为1,则证明,可能存在 $P_j$ 故障。\\ 
\indent 通过$R^T$与$T$进行逻辑乘,得到$T'$,即反映了所有错误信息的可能原因。 将$T'$与之前定义的$F$进行对应位置的与运算,筛选错误原因,即可得到电网的故障结果向量$F'$。

\begin{figure}[htbp]
    \centering
    \includegraphics[width=10cm]{zrk-3.png}
    \caption{算法流程}
\end{figure}

\subsubsection{算法举例}
简单的配电系统录像图如图\ref{zrk-1}所示。该线路继电保护系统由三段式电流保护$OR$,和断路器$CB$组成。假设$L_1$上存在故障,组保护$OR_1$将动作使断路器$CB_1$跳闸。若此时断路器拒动,则后备保护$OR_3$将动作并使断路器$CB_3$跳闸从而将故障隔离。依据三个基本节点的定义,故障元件节点、保护节点、断路器动作节点数量分别为3个,3个,5个。通过上述过程,可以建立如图\ref{zrk-2}所示电网因果网络模型和如表\ref{zrk-3}所示规则矩阵$R$\\
\indent 故障时,各设备动作如下:
\begin{enumerate}[itemindent=4em]
    \item 保护动作:$OR_2 , OR_3$
    \item 断路器跳闸:$CB_3$
    \item 断路器拒动:$CB_2$
\end{enumerate}
依据故障时各元件动作,易知$P_5 , P_6 , P_8 , P_{10}$为真 ,则$T = [0\,0\,0\,0\,1\,1\,0\,1\,0\,1\,0]$,由实际警告信息得$ F = [0\,1\,0\,0\,0\,0\,0\,1\,0\,0\,0\,1]$ , 进行逆向推理计算得$T' = R^T \bigodot  T$,通过$F' = T' \bigwedge F$进行筛选得$F' =  [0\,0\,0\,0\,0\,0\,1\,0\,0\,0\,1]$。有$F'$可知,故障元件可能为$L_1$ 或 $BUS$

\subsubsection{配电网信息差异故障检测差异}
\indent 基于上述例子中的规则矩阵,当电网的警告信息出现差异,即$CB_2$的跳闸结果丢失时,各设备动作如下:
\begin{enumerate}[itemindent=4em]
    \item 保护动作:$OR_2 , OR_3$
    \item 断路器跳闸:$CB_3$
\end{enumerate}
得$T = [0\,0\,0\,0\,1\,1\,0\,1\,0\,0\,0] $, $F = [0\,1\,0\,0\,0\,0\,0\,1\,0\,0\,0\,1]$ , 依据算法,计算出最终错误信息向量$F' = [0\,0\,0\,0\,0\,0\,1\,0\,0\,0\,1]$。由$F'$可知,故障元件可能为$L_2$ 和 $BUS$ , 检测结果依然正确

\indent 基于上述例子中的规则矩阵,当电网的警告信息出现差异,即$OR_2$的保护信息丢失时,各设备动作如下:
\begin{enumerate}[itemindent=4em]
    \item 保护动作:$OR_3$
    \item 断路器跳闸:$CB_3$
    \item 断路器拒动:$CB_2$
\end{enumerate}
得$T = [0\,0\,0\,0\,1\,1\,0\,0\,0\,1\,0] $, $F = [0\,1\,0\,0\,0\,0\,0\,1\,0\,0\,0\,1]$ , 依据算法,计算出最终错误信息向量$F' = [0\,0\,0\,0\,0\,0\,0\,0\,0\,0\,1]$。由$F'$可知,故障元件可能为$BUS$ , 检测结果出现偏差

\subsection{问题三}
基于动态模式以及时空角度的拓扑结构,自然需要软自愈方案来实现配电网的
故障方案解决,并且提前判断故障征兆,从而实现稳定供电;此外,通过软自愈,
能够通过实时检测,优化负荷分配,实现负荷转移。
\subsubsection{建立抽象拓扑结构}
将配电网中的联络开关看作无向边$e$,从而得到所有边的集合$E$;
根据开关的开闭状态将$E$分为$E1$以及$E2$,分别表示处于开状态的开关
以及闭状态的边,两个集合的元素可以动态转化。在进行故障判断、
故障预知以及负荷调整时,给予边$e$一个权值$Cost_m(I,t)$,表示$t$时刻
某边$e_m$的电流值为$I$,从而实现基于连续时空拓扑信息+电流的判断。
之后将配电网的开关之间的馈线看作顶点,组成顶点集$V$,得到$G = (V,E)$ 
的无向图。\\
\indent 对于题目中图二进行抽象拓扑得到的图,如图8所示:
\begin{figure}[htbp]
    \centering
    \includegraphics[width=15cm]{9.png}
    \caption{抽象化样例图}
\end{figure}


\subsubsection{故障判断}
需要通过电流值变化判断故障点,所以我们根据中压配电网的常规电流值
给出故障电流阈值$I_{bound}$以及优化电流阈值$I_{PreBound}$。如果超过$I_{bound}$则
说明该开关附近已经发生了线路故障,造成了电流异常; 如果超过$I_{PreBound}$则说明
附近线路负载过大,需要进行负载平衡优化。根据中压配电网特性设置\cite{yy6} $I_{bound} = 800 A$,
$I_{PreBound} = 600 A$,从而根据阈值即可判断出异常的开关,
而开关不会对电流产生影响,如果有故障馈线,
则开关两侧的线路可能存在故障点,即可以得出结论:
$$
\text{ {\it 若馈线存在故障} }\Longleftrightarrow \text{ {\it 馈线所直连的所有开关电流值都异常}}
$$
\indent 所以我们根据图$G$异常电流的边,自然确定出公共点,从而定位到出问题的馈线位置。




\subsubsection{故障隔离}
实现故障判断后,进行开关动作进而完成故障隔离。实现故障隔离要关闭开关。
容易得出结论:{\bf {\it 关闭故障点所邻接的开关使得故障范围最小。}}
基于此,对于图$G$只需要把出问题的点$J$所有的边$e_j$都从$E1$删除转移到$E2$中,
即可实现故障点的孤岛化处理以及故障隔离。
\subsubsection{恢复供电}
因为故障隔离的开关动作会导致除故障点之外的孤岛产生,
判断方法是:$$ \text { {\it 连通分支数目 > 变电站个数 + 故障点个数} $\Longleftrightarrow$  {\it 存在额外无故障段孤岛}} $$
\indent 恢复供电就是要将此类节点并入无故障的电网,
从而变成正常的电网。通过使用$BFS$算法以及标号的思想,
将每次从电源点BFS到的所有点都标一个相同的号,从而得出所有的连通分支
;进而遍历孤岛节点的边,选择添加未合上并且能够并入带有电源的连通分支
的开关(边),即实现了恢复供电!题目所给的例图的解决方案如图9: 
\begin{figure}[htbp]
    \centering
    \includegraphics[width=16cm]{10.png}
    \caption{抽象化样例图}
\end{figure}

\subsubsection{优化及负载均衡}
由于软自愈系统的计算能力远超过硬自愈系统,
所以除了电网故障自愈还能实现线路正常状态下的运行优化,
即负载均衡。当一部分供电网络负载过大,很可能出现故障,
所以把负载过大的地区转移到负载较小的网络,就是实现负载均衡。\\
\indent 因为配电网电压恒定,所以功率$P \propto I$,故仍可以通过电流过大判断是否要进行负载调整。
根据故障判断中的$I_{PreBound}$使用同样的算法可以定位到负载较大的馈线段。之后,通过BFS从需要优化的节点开始扩展到最远叶节点,然后
对此叶节点进行负荷转移,即通过开关的状态变化更换供电站。 

\subsubsection{成本差异估计}
"硬自愈"是依靠专用电子设备装置接受系统的实时监测数据,通过自动化系统(硬件),从而实现线路开关的控制完成故障自愈。\\
\indent 软自愈是通过计算机系统接受实时监测数据,通过软件算法完成线路故障控制的推测,然后再发出开关动作指令。还能够实现线路优化、负载均衡、提前预测等功能。
\indent 成本差异体现在:
\begin{enumerate}
    \item 设备价格:硬自愈需要更多的成本去铺设基础设施,软自愈则可以通过仿真计算减少成本。
    \item 维护成本:硬自愈需要日常维护硬件设施的完整程度,才能保证运行正常,而软自愈则不需要。
    \item 用户体验:
\end{enumerate}

\section{模型评价与推广}
\subsection{模型的综合评价}

{\bf 问题一}:
     指标清晰,对于可靠性评估目标进行了具体的量化分析;
     所有问题都基于了配电网拓扑信息,并且可以通过等值法简化配电网,简化了问题,便于统一模型。
    从PDO角度出发,关注一段时间之内的配电网可靠性评估。\\
\indent {\bf 问题二}:由于多种故障情形都可能得到相同得警告信息。例如在计算举例中,当L1故障,OR1执行保护动作,CB1拒跳,导致OR3执行保护动作,CB3跳闸。
当L1与BUS同时发生故障时,可能会得到相同的警告信息,导致计算的故障结果存在冗余。\\
\indent {\bf 问题三}:此模型能根据开关的电流值,根据电流值是否超出阈值,判断电网中是否发生跳闸故障。并且根据电网中的负载量,自动重构电网,以降低故障发生的可能性。
再电网故障判断的过程中,依赖于各个开关的电流输入,以判断开关是否跳闸,对于开关据动等情况没有充分的考虑。
对于电网重构优化,没有过考虑对于负载最优的重新分配。当电网中的某一条线路中的过大的负载,自动将其分配到其他的新的线路中,但未分配结果的最优性。

 



\subsection{模型的改进}
{\bf 问题一}:还需要对具体不同元件等效到最小路的方法进行区分,才能更精确地评估出可靠性。\\
\indent {\bf 问题二}:由于在同一通路上,两点同时发生故障的概率较低,依据故障点的拓扑信息,进行溯源,得到最源头的故障原因,减少结果的冗余。\\
\indent {\bf 问题三}:根据周围多个监测点的电流值,充分考虑开关据动和元件损坏等情况,提高系统的可靠性,降低数据变量的灵敏度。

\section{参考文献}
\begin{thebibliography}{9}%宽度9
    \bibitem[1]{yy5}
    文向南. 基于因果网络的配电网故障诊断研究[D].青岛大学,2017.
    \bibitem[2]{yy1}
    赵华,王主丁,谢开贵,李文沅.中压配电网可靠性评估方法的比较研究[J].电网技术,2013,37(11):3295-3302.
    \bibitem[3]{yy2}
    别朝红,王秀丽,王锡凡.复杂配电系统的可靠性评估[J].西安交通大学学报,2000(08):9-13.
    \bibitem[4]{yy3}
    Billintoo R, Wang P. Reliability-network-equivalent approach to distribution-system-reliability evalation [J].IEEE
    Proc,1998,145(2):149~ 153.
    \bibitem[5]{yy4}
    高亚静,林琳,刘建鹏.油田配电网的可靠性与灵敏度分析[J].电力科学与工程,2013,29(12):36-40.
    \bibitem[6]{yy6}
    \url{http://www.abcbxw.com/news/8139.html} 黔西南州人民政府网—兴义供电局2018年第一批10kv线路
\end{thebibliography}

\section{附录}
\begin{appendices}
\noindent {\bf 问题一}:计算C++程序:
\begin{lstlisting}[language=C]
#include<bits/stdc++.h>
using namespace std;

/*假设一般规模的配电网图*/
const int N=100; int n;const int INF=100000;
int p[N][N],d[N],path[N];//path数组用于输出路径
map<int,string> road; //记录路径


//Input datas: 
/*----------------------------------------------------*/
float probably[N][N];//等效故障率
float Outage_durations[N][N];//等效每次故障平均停电持续时间
float users[N];//负荷点的用户数(K)
bool Lps[N];//是不是负荷点
/*----------------------------------------------------*/


//Output datas:
/*----------------------------------------------------*/
    //负荷点:
    float Lp_break_proba[N];//负荷点故障率
    float Lp_break_times[N];//负荷点故障持续时间

    //系统平均停电持续时间(SAIDI) 平均停电频率(SAIFI)
    //系统平均供电可用率(ASAI)  平均供电不可用率(ASUI)
    float SAIDI=0,SAIFI=0,ASAI = 0,ASUI = 0;
/*----------------------------------------------------*/


//最短路程序
void dijkstra(int sec,int n)//sec为出发节点,n表示图中节点总数
{
    int i,j,min,min_num;
    int vis[N]={0,};
    for(i=0;i<n;i++)
    {
        d[i]=p[sec][i];
    }
    vis[sec]=1;d[sec]=0;
    for(i=1;i<n;i++){
        min=INF;
        for(j=0;j<n;j++)
        {
            if(!vis[j]&&d[j]<min)
            {
                min=d[j];
                min_num=j;
            }
        }
        vis[min_num]=1;
        for(j=0;j<n;j++)
        {
            if(d[j]>min+p[min_num][j])
            {
                path[j]=min_num;
                d[j]=min+p[min_num][j];
            }
        }
    }
}

//打印路径
void print(int sec,int n)//sec为出发节点,n表示图中节点总数
{
    int i,j;
    //由于记录的中途节点是倒序的,所以使用栈,获得正序
    stack<int> q; 
    for(i=1;i<n;i++)  
    {//打印从出发节点到各节点的最短距离和经过的路径
        j=i;
        while(path[j]!=-1)      //如果j有中途节点
        {
            q.push(j);          //将j压入堆
            j=path[j];          //将j的前个中途节点赋给j
        }
        q.push(j);
        printf("%d=>%d, length:%d, path: %d ",sec,i,d[i],sec);
        while(!q.empty())       //先进后出,获得正序
        {
            printf("%d ",q.top());//打印堆的头节点
            q.pop();            //将堆的头节点弹出
        }
        printf("\n");
    }
}

void calculate_lp(){
    //计算 各负荷点的可靠性指标

    int pre=0,now;
    for(int i =1;i<=n;i++){
        for(auto v:road[i]){
            if(v == ' ' ) continue;
            else if(v == '0') continue;
            else {
                now = v - '0';
                Lp_break_proba[i] += probably[pre][now];
                Lp_break_times[i] += probably[pre][now]*
                                     Outage_durations[pre][now];
                pre = now;
            }
        }
        pre = 0;
    }
    
    for(int i =1;i<=N;i++){
        if(Lps[i]) {
            cout<<"V"<<i<<"负荷点平均故障率: ";
            printf("%.2f",Lp_break_proba[i]);
            cout<<" 平均故障时间: ";
            printf("%.2f\n",Lp_break_times[i]);
        }
    }
}

void calculate_sys(){
    //计算 系统的的可靠性指标
    
    float users_num = 0;
    for(int i=1;i<=N;i++){
        if(Lps[i]){
            users_num += users[i];
            SAIDI = Lp_break_times[i]*users[i];
            SAIFI = Lp_break_proba[i]*users[i];
        }
    }

    ASAI = 1 - (SAIDI/(users_num*8760));
    ASUI = 1 -ASAI;
    SAIDI = SAIDI / users_num;
    SAIFI = SAIFI / users_num;
    cout<<"ASAI /%: "<<ASAI*100<<"\n";
    cout<<"ASUI /%: "<<ASUI*100<<"\n";
    cout<<"SAIDI/(h/(户·a)): "<<SAIDI<<"\n";
    cout<<"SAIFI/(次/(户·a)): "<<SAIFI<<"\n";
}

int main()
{
    memset(path,-1,sizeof(path));//将path数组初始化为-1
    int i,j; n=9;//点个数
    for(i=0;i<n;i++){//初始化
        for(j=0;j<n;j++){
            p[i][j]=(i==j?0:INF);
        }
    }
    
//Graph:
    //V0 是电源点  V2 V6 V7 V8是负荷点a b c d  
    Lps[2]=Lps[6]=Lps[7]=Lps[8]=true;
    //其余是 主馈线和分支线的交点 用来分隔电路
    p[0][1]=1;p[1][2]=1;p[1][3]=1;p[3][4]=1;p[4][5]=1;
    p[3][6]=1;p[4][7]=1;p[5][8]=1;
    dijkstra(0,n); //求从节点0出发到各节点的最短距离
    print(0,n);    //打印从节点0出发到各节点的最短距离和路径
   
    
//Input datas:
    //假设的供电线的可靠性参数
    //等效故障率:
    probably[0][1]=0.2;probably[1][2]=0.1;probably[1][3]=0.15;
    probably[3][4]=0.2;probably[4][5]=0.25;probably[3][6]=0.2;
    probably[4][7]=0.2;probably[5][8]=0.1;
    //故障停电时间
    Outage_durations[0][1]=2;Outage_durations[1][2]=5;
    Outage_durations[1][3]=5;Outage_durations[3][4]=5;
    Outage_durations[4][5]=3;Outage_durations[3][6]=3;
    Outage_durations[4][7]=2;Outage_durations[5][8]=1;
    //用户数
    users[2] = 3;users[6]= 4;users[7]=5;users[8]=5;
    //已知最短路的情况下进行计算:
    road[1] = "0 1";         road[2] = "0 1 2";
    road[3] = "0 1 3";       road[4] = "0 1 3 4";
    road[5] = "0 1 3 4 5";   road[6] = "0 1 3 6";
    road[7] = "0 1 3 4 7";   road[8] = "0 1 3 4 5 8";


//Output datas:
    calculate_lp();
    calculate_sys();
    
    return 0;
}
\end{lstlisting}

\noindent {\bf 问题二}: 故障检测函数
\begin{lstlisting}[language=matlab]
    function [F1] = TopoErrorDetect(R,T,F)
    %Inputs:
        %R:规则矩阵
        %T:警告信息向量
        %F:可能故障元件向量
            %Outputs:
        %F1:检测结果
    T1 = logical(R' * T');%依据输入信息,逆推可能各章的元件
    F1 = logical(F' .* T1);%依据实际,筛选故障元件
    end
\end{lstlisting}  

\begin{lstlisting}[language=C]
    #include<bits/stdc++.h>
#include<algorithm>
#include"queue"

using namespace std;
struct Edge{
    int u,v,w,nxt;
    bool open;//open代表开关是否打开
};
class elecNet{
public:
    Edge * Edges;
    int* head;
    int tot;
    int iBound;//电流上界,超出界限则认为发生故障,下一步将进行跳闸.
    int iPreBound;
    vector<int> lable;
    int s1 ;
    int s2;
    int n;
    int m;
    vector<int> brokenNode;
    vector<int> vis;
    vector<int> alwaysClose;
    vector<int> overPre;
    vector<int> overPre_1;
    vector<int> overPre_2;
    vector<int > removeNodes;
    void findRemoveNode();
    void remove();
    elecNet(int n0,int m0,int sa,int sb) {
         n = n0;
         m = m0;
        vis = vector<int>(n+1,0);
        Edges = new Edge[2*m+2];
        head  =  new int [n+1];
        lable = vector<int>(n+1,0);
        lable[sa] = 1;
        lable[sb] = 2;
        s1 = sa;
        s2 = sb;
        tot = 0;
        memset(head , sizeof(head), 0);
        memset(Edges, sizeof(Edges),0);
        for(int i = 0;i <=2*m+1;i++ )
            Edges[i].nxt = 0;
        for(int i = 0;i <= n;i++)
            head[i] = 0;
        iBound = 10;
        iPreBound = 8;
    }

    void addEdge(int u,int v,bool open);//初始化
    void addAlwaysClose(vector<int> a);
    void inputEdge(vector<int>&);//每一个时刻重新输入电流
    void connect();
    void bfs(int s,int flag);
};
void elecNet:: addEdge(int u,int v,bool open){
    tot++;
    Edges[tot].u = u;
    Edges[tot].v = v;
   // Edges[tot].w = w;
    Edges[tot].open = open;
    Edges[tot].nxt = head[u];
    head[u] = tot;
}

void elecNet::inputEdge(vector<int> & a) {
    vector<int> preBroken;
    overPre_1 =  vector<int>() ;
    overPre_2 =  vector<int>() ;
    brokenNode = vector<int>();
    for(int j = 1;j<= m;j++){

        int i = j*2;
        Edges[i].w = a[j];
        Edges[i-1].w = a[j];
        if(a[j] > iBound) {//电流
            Edges[i].open = false;//关闭电闸
            Edges[i-1].open = false;//关闭电闸
            cout << "switch " << j <<": close"<<endl;
            if (find(preBroken.begin(), preBroken.end(),
            Edges[i].u) != preBroken.end())
                brokenNode.emplace_back((Edges[i].u));
            else preBroken.emplace_back(Edges[i].u);
            if (find(preBroken.begin(), preBroken.end(),
            Edges[i].v) != preBroken.end())
                brokenNode.emplace_back((Edges[i].v));
            else preBroken.emplace_back(Edges[i].v);
        }
        else if(Edges[i].w > iPreBound){
            if(!Edges[i].open)continue;
           if(find(overPre_1.begin(),overPre_1.end(),
           Edges[i].u) != preBroken.end())
               overPre_2.emplace_back(Edges[i].u);
           else overPre_1.emplace_back(Edges[i].u);
            if(find(overPre_1.begin(),overPre_1.end(),
            Edges[i].v) != preBroken.end())
                overPre_2.emplace_back(Edges[i].v);
            else overPre_1.emplace_back(Edges[i].v);
            for(int k = 0;k < overPre_1.size();k++){
                if(find(overPre_2.begin(),overPre_2.end(),
                overPre_1[k]) == overPre_2.end())
                    overPre.emplace_back(overPre_1[k]);
            }
        }
    }
    for(int i = 0;i < brokenNode.size();i++)
    {  cout <<"broken node : "<<brokenNode[i]<<endl;
        alwaysClose.emplace_back(brokenNode[i]);
    }
    return ;
}

void elecNet::connect() {


   for(int i = 0;i <= lable.size();i++)
        lable[i]  = 0;
    lable[s1] = 1;
    lable[s2] = 2;
    for(int i = 0; i< vis.size();i++)
        vis[i] = 0;
    bfs(s1, 1);
    bfs(s2, 2);
    list<int> island;
    for(int i=1;i <= n; i++){
        if(!lable[i]) {
            if (find(brokenNode.begin(), brokenNode.end(), i)
             == brokenNode.end())
                island.push_back(i);
        }

    }
    bool flag = true;
    while(flag){
        flag = false;
        for(auto it = island.begin();it != island.end(); it++){
            int u = *it;
            if(find(alwaysClose.begin(),alwaysClose.end(),u)
            != alwaysClose.end()) continue;
            for(int i = head[u];i;i = Edges[i].nxt){
                int v = Edges[i].v;
                if(lable[v]){
                    lable[u] = lable[v];//并入电网
                    Edges[(i+1)/2*2].open = true;//打开电闸
                    Edges[(i+1)/2*2-1].open = true;
                    cout << "switch " << (i+1)/2 <<": open"
                    <<endl;
                    it = island.erase(it);//从孤岛中删除
                    flag = true;
                    break;
                }

            }
        }
    }
    return ;
}

void elecNet::bfs(int s,int flag) {
    queue<int> q;
    lable[s] = flag;
    q.push(s);
    vis[s] = 1;
    while(!q.empty()){
        int u = q.front();
        q.pop();
        for(int i = head[u];i;i = Edges[i].nxt){
            int v = Edges[i].v;
            if(vis[v])continue;
            if(lable[v] == 0 && Edges[i].open){
                lable[v] = flag;
                q.push(v);
                vis[v] = 1;
            }
        }
    }
    return ;
}

void elecNet::addAlwaysClose(vector<int> a) {
    alwaysClose = vector<int>(a);
}

void elecNet::findRemoveNode() {
    for(int i = 0;i < overPre.size();i++){
        queue<int > q;
        vector<int> vis(n,0);
        q.push(overPre[i]);
        vis[overPre[i]];
        while(!q.empty()){
            int u = q.front();
            q.pop();
            int cnt = 0;
            for(int e = head[u];e;e = Edges[e].nxt){
                int v = Edges[e].v;
                if(find(overPre_2.begin(),overPre_2.end(),v)
                 != overPre_2.end()) continue;
                if(!Edges[e].open) continue;
                cnt++;
                q.push(v);
            }
            if(cnt == 0){
                removeNodes.emplace_back(u);
                break;
            }

        }

    }
}

void elecNet::remove() {
    for(int i = 0;i < removeNodes.size();i++){
        int u = removeNodes[i];
        int u1 = u;
        int eClose = 0;
        for(int e = head[u]; e; e = Edges[e].nxt){
            if(Edges[e].open) {
                u1 = Edges[e].v;
                eClose = e;
                continue;
            }
        }
        for(int e = head[u];e;e = Edges[e].nxt){
            if(Edges[e].open) continue;
            int v = Edges[e].v;
            if(lable[v] == lable[u] || lable[v] == 0)
             continue;
            //并到另一个树上
            int j = (eClose+1)/2*2;
            Edges[j].open = false;
            Edges[j-1].open = false;
            cout <<"cut node "<< u << " from "<<lable[u]
            <<" to "<<lable[v]<<endl;
            cout <<"switch "<<j/2 << ": close"<<endl;
            int k = (e+1)/2*2;
            Edges[k].open = true;
            Edges[k-1].open = true;
            cout <<"switch " <<k/2 <<": open"<<endl;
            lable[u] = lable[v];
            break;
        }

    }
}


int main() {
     freopen("a.in","r",stdin);
    freopen("a.out","w",stdout);
    int n0,m0,s1,s2;
    cin >>n0>>m0>>s1>>s2;
    elecNet en(n0,m0,s1,s2);
    int t;
    cin >>t;
    vector<int> close;
    for(int i = 0; i < t;i ++){
        int tmp;
        cin >> tmp;
        close.emplace_back(tmp);
    }
    en.addAlwaysClose(close);
    for(int i = 1;i <= m0;i++){
        int u,v;
        bool open;
        cin >> u>>v>>open;
        en.addEdge(u,v,open);
        en.addEdge(v,u,open);
    }

    vector<int> a;
    a.emplace_back(0);
    for(int i = 1;i <= m0;i++){
        int w;
        cin >> w;
       a.emplace_back(w);
     //  a.emplace_back(w);
    }
    en.inputEdge(a);
    en.connect();
    en.findRemoveNode();
    en.remove();
    return 0;
}
//input:
//14 13 1 2
//4 5 7 11 14
//1 3 1
//3 4 1
//4 5 0
//4 6 1
//6 8 1
//6 7 0
//8 9 0
//9 10 1
//10 12 1
//10 11 0
//12 14 0
//12 13 1
//2 13 1
//1
//2
//0
//11
//11
//0
//0
//2
//2
//0
//0
//2
//3
\end{lstlisting}

\end{appendices}

\end{document} 